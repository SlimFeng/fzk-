\documentclass[12pt]{exam}

% Packages
\usepackage{amsmath}
\usepackage{amsfonts}
\usepackage{amssymb}
\usepackage{graphicx}
\usepackage{url}
\usepackage{enumitem}
\usepackage[margin=1in]{geometry}

% Header and footer
\pagestyle{headandfoot}
\firstpageheader{\textbf{Electromagnetism Final Exam}}{}{\textbf{Date:} \underline{\hspace{2cm}}}
\runningheader{Electromagnetism Final Exam}{}{\textbf{Page \thepage}}
\runningheadrule
\firstpagefooter{}{}{}
\runningfooter{}{}{}

% Exam information
\title{Electromagnetism Final Exam}
\author{}
\date{}

% Start the document
\begin{document}

\maketitle

% Instructions
\begin{center}
\textbf{Instructions}
\end{center}
\begin{itemize}[label=$\bullet$,leftmargin=*]
    \item This exam has \numquestions{} questions and is worth \numpoints{} points. The total time allowed is \numpages{} pages.
    \item Show all work and provide clear explanations for your answers.
    \item Write your answers in the space provided. If you need more space, use the back of the page and clearly indicate where your answer can be found.
    \item You may use a calculator and a formula sheet. Calculators must be cleared of all data before the start of the exam.
    \item Cheating will not be tolerated and will result in a grade of zero on the exam and possible further disciplinary action.
\end{itemize}

% Questions
\begin{questions}

% Question 1
\question[10] A point charge $q$ is located at a distance $d$ from an infinite plane conductor. Determine the electric field at a point $P$ located a distance $h$ above the plane and directly above the charge.

\vspace{8cm}

% Question 2
\question[15] A long, thin cylindrical shell with radius $a$ carries a uniformly distributed total charge $Q$. A point charge $q$ is located at a distance $r$ from the axis of the shell.
\begin{parts}
    \part Determine the electric field at a point inside the shell, at a distance $s$ from the axis, where $s < a$.
    \vspace{8cm}
    \part Determine the electric field at a point outside the shell, at a distance $s$ from the axis, where $s > a$.
    \vspace{8cm}
\end{parts}

% Question 3
\question[10] Two parallel plates carry opposite charges of magnitude $Q$. The plates are separated by a distance $d$. Find the potential difference between the plates and the electric field between them.

\vspace{8cm}

% Question 4
\question[15] A long straight wire carries a current $I$ and is surrounded by a cylindrical surface of radius $r$. Use Ampere's law to determine an expression for the magnetic field magnitude as a function of distance $r$ from the wire.

\vspace{8cm}

% Question 5
\question[20] A solenoid is made by tightly wrapping $n$ turns of wire of radius $a$ around a long cylindrical core of radius $R$, where $a \ll R$. The wire is wrapped in a way that the current flows around the solenoid in the direction of the solenoid's axis.
\begin{parts}
    \part Determine the magnetic field inside the solenoid as a function of distance from the axis.
    \vspace{8cm}
    \part Determine the magnetic field outside the solenoid.
    \vspace{8cm}
\end{parts}

% Question 6
\question[10] A particle of charge $q$ and mass $m$ moves in a uniform magnetic field $\vec{B}$ with velocity $\vec{v}$. Show that the motion of the particle is circular.

\vspace{8cm}

% Question 7
\question[10] A circular loop of wire with radius $a$ lies in the $x$-$y$ plane with its center at the origin. The loop carries a current of magnitude $I$. Find the magnitude and direction of the magnetic field at the origin.

\vspace{8cm}

% Question 8
\question[10] Consider a capacitor with two parallel plates of area $A$ and separation $d$. The plates are charged with a charge $Q$ and $-Q$. Calculate the energy stored in the electric field between the plates.

\vspace{8cm}

% End of the exam
\end{questions}

\end{document}
